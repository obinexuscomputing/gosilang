\documentclass{article}
\usepackage{amsmath}
\usepackage{amsthm}
\usepackage{amssymb}

\title{Automaton State Minimization and AST Optimization}
\author{Nnamdi Michale Okpala}
\date{November 12 2024}

\begin{document}
\maketitle

\section{Layman's Explanation}
Automaton state minimization is about taking a finite state machine (FSM)---which may have redundant states---and simplifying it to use the smallest number of states possible, while still performing the same behavior. This concept is particularly relevant when associated with minimizing the \textbf{abstract syntax tree (AST)}, which is a tree representation of the machine's rules or transitions.

\section{Formal Definition}

\subsection{Automaton Representation}
Let the automaton $A$ be represented as a 5-tuple:
\[
A = (Q, \Sigma, \delta, q_0, F)
\]
where:
\begin{itemize}
    \item $Q$: Finite set of states
    \item $\Sigma$: Finite alphabet (input symbols)
    \item $\delta: Q \times \Sigma \to Q$: Transition function
    \item $q_0 \in Q$: Initial state
    \item $F \subseteq Q$: Set of accepting (final) states
\end{itemize}

\subsection{Abstract Syntax Tree (AST)}
The AST represents the structure of transitions and states in the automaton as a tree. Each node corresponds to:
\begin{itemize}
    \item A state $q \in Q$
    \item An input $\sigma \in \Sigma$ or output transition $\delta(q, \sigma)$
\end{itemize}

\subsection{State Equivalence}
Define two states $p, q \in Q$ as \textbf{equivalent} ($p \sim q$) if for every possible input sequence $w \in \Sigma^*$, the automaton starting at $p$ and $q$ ends in the same type of state (both accepting or both non-accepting).

Mathematically:
\[
p \sim q \iff \forall w \in \Sigma^*, \delta^*(p, w) \in F \iff \delta^*(q, w) \in F
\]

Here, $\delta^*$ is the extended transition function for sequences:
\[
\delta^*(q, \epsilon) = q, \quad \delta^*(q, a w) = \delta^*(\delta(q, a), w)
\]

\subsection{Minimization}
The minimization process constructs a new automaton $A' = (Q', \Sigma, \delta', q_0', F')$ where:
\begin{itemize}
    \item $Q'$: Partition of $Q$ into equivalence classes under $\sim$
    \item $\delta'(C, a) = [\delta(q, a)]$ for any $q \in C$, where $C$ is an equivalence class
    \item $q_0' = [q_0]$ (the equivalence class containing the initial state)
    \item $F' = \{C \in Q' \mid C \cap F \neq \emptyset\}$
\end{itemize}

\section{Summary}
To minimize an automaton:
\begin{enumerate}
    \item Identify equivalent states using the equivalence relation $\sim$
    \item Partition states into equivalence classes
    \item Construct the reduced automaton using the partitions
    \item Reflect the minimized structure in the AST by reducing nodes and transitions accordingly
\end{enumerate}

\end{document}